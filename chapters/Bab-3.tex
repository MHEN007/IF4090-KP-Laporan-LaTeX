\chapter{$<$Judul Topik KP$>$}
\textit{Pada bagian ini dijelaskan mengenai pelaksanaan Kerja Praktek meliputi deskripsi persoalan, proses dan pencapaian hasil. Lebih rinci terkait Dokumen Teknis, bisa mengacu pada Lampiran Dokumen Teknis. Jika tidak disertai Dokumen Teknis HARUS melampirkan pernyataan bahwa Dokumen Teknis tersebut bersifat confidential dengan ditandatangani oleh Pembimbing atau Penanggung Jawab di Perusahaan (menggunakan formulir yang sudah disediakan).
}

\textit{Pencapaian Hasil yang dilaporkan harus sampai pada evaluasi atau umpan balik dari perusahaan (misal apakah sudah di coba di perusahaan, ataukah sudah memberikan training kepada calon pengguna di perusahaan, apakah ada tindak lanjut yang harus dilakukan untuk perbaikan, dan sebagainya).
}

\textit{Dalam bab ini dituliskan kesulitan atau kemudahan yang ditemui selama kerja praktek terkait dengan komunikasi antar personal (dengan lingkungan kerja), kerja tim, pengetahuan yang mendukung pelaksanaan KP, serta bagaimana solusi untuk kesulitan yang dihadapi.}

\section{Deskripsi/Analisis persoalan sesuai dengan topik KP}
\textit{Tuliskan analisis dari persoalan yang harus diselesaikan dalam kerja praktek, termasuk usulan solusi untuk persoalan tersebut. Termasuk di dalamnya latar belakang persoalan, deskripsi persoalan, batasan yang harus diselesaikan dalam KP, teknologi terkait yang mendukung solusi dari persoalan, dan hal lain yang terkait dengan persoalan KP. Berikut contoh sitasi \cite{aws_ocr}}

\section{Beri judul sub bab sesuai dengan proses pelaksanaan KP}
\textit{Uraikanlah proses yang dikerjakan selama Kerja Praktek termasuk hambatan yang ditemui dan cara penyelesaian jika ada. Dalam bagian ini juga dituliskan kakas atau pengetahuan yang dimanfaatkan dalam proses pelaksanaan KP.}

\section{Beri judul sub bab sesuai dengan proses pelaksanaan KP}
\textit{Uraikan berbagai hasil yang diperoleh selama Kerja Praktek, rinciannya mengacu pada lampiran dokumen teknik jika ada. Hasil selama KP dikaitkan juga dengan tujuan KP di sub bab I.3.}